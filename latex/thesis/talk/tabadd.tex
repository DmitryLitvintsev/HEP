%
%
%
%   Macros to produce Bar-Diagramms in Tabels
%
%
\newdimen\tabbarunitwidth
\newbox\tabvoidbox
\newdimen\tabvoiddim
\tabbarunitwidth=1cm
\countdef\mystart=255
\newdimen\tabbarunitheight
\tabbarunitheight=1ex
\newcounter{fulltabnumber}
\setcounter{fulltabnumber}{0}
\def\skipcolsep{\kern-\tabcolsep}
\def\addcolsep{\kern\tabcolsep}
\def\ttabb{&}
%
%
% usage: \tabful{entry}
% fills 'entry' into the table discarding the sep[eration at the tab columns
%
\def\tabful#1{\nothing\skipcolsep{#1}\skipcolsep\nothing}
%
% usage \tabbox{width}{content}
% put entry #2 into a parbox of width #1
%
\def\tabbox#1#2{%
\skipcolsep\parbox[t]{#1}{#2}\skipcolsep}
%
%
\def\tablist#1#2#3#4{\skipcolsep\parbox[t]{#1}{#2}&\parbox[t]{#3}{#4}\skipcolsep}
%
%
%   Produce a multicolumn entry of #1 tabs width with format #2 and contents #3
%   without left and right tabcol spacing
%   usage like multicolumn:
%        \fullmlti{n}{|c...}{the entry}
%
\def\fullmulti#1#2#3{%
\multicolumn{#1}{#2}{\skipcolsep\makebox[\tabbarunitwidth]{#3}\skipcolsep}}
%
%   usage: \fulltab{the entry}
%   puts the entry centered in the table with an entry length of \tabbarunitwidth
%   discarding the seperation at the tab-columns
%
\def\fulltab#1{\skipcolsep\makebox[\tabbarunitwidth]{#1}\skipcolsep}
%
%   usage: \fullbar
%   puts a bar in the table with a length of \tabbarunitwidth and height of
%   \tabbarunitheight discarding the seperation at the tab-columns
%
\def\fullbar{\skipcolsep\rule{\tabbarunitwidth}{\tabbarunitheight}\skipcolsep}
%
%   usage: \tabbar{n}
%   puts a bar in the table as n entries with a length of n*\tabbarunitwidth and
%   height of \tabbarunitheight discarding the seperation at the tab-columns
%
\def\tabbar#1{\mystart=#1%
\begingroup%
\loop\ifnum\mystart>0%
\aftergroup\fullbar%
\ifnum\mystart>1\aftergroup\ttabb\fi%
\advance\mystart by-1\repeat%
\endgroup}
%
%   usage \ntab{n}  producse n times &
%
\def\ntab#1{\mystart=#1%
\begingroup%
\loop\ifnum\mystart>0\aftergroup\ttabb\advance\mystart by-1\repeat%
\endgroup}
%
%   usage \tabbarnums{start}{n}
%   puts a row of n entries with increasing numbers starting with start in the table,
%   all equally spaced with \tabbarunitwidth discarding any tab columun spacing
%
\def\fulltabnum{\fulltab{\arabic{fulltabnumber}}\addtocounter{fulltabnumber}{1}}
\def\tabbarnums#1#2{\mystart=#2\setcounter{fulltabnumber}{#1}%
\begingroup%
\loop\ifnum\mystart>0%
\aftergroup\fulltabnum%
\ifnum\mystart>1\aftergroup\ttabb\fi%
\advance\mystart by-1\repeat%
\endgroup}
%
%   \septab{text} adds column seperation in front of and after text
%   \sepbtab adds column seperation in front of till tab following text
%   \sepatab adds column seperation in after till tab following text (just before tab)
%
\def\septab#1{\addcolsep #1\addcolsep}
\def\sepbtab#1&{\addcolsep #1&}
\def\sepatab#1&{#1\addcolsep&}
%
%   usage: \widthtab{width}{the entry}
%   puts the entry centered in the table with an entry length of width
%   discarding the seperation at the tab-columns
%
\def\widthtab#1#2{\skipcolsep\makebox[#1]{#2}\skipcolsep}
%
%
% usage: \toplinespace{space}  adds extra space st top of line
%        \botlinespace{space}  adds extra space at bottom of line
%
%
\def\toplinespace#1{%
\setbox\tabvoidbox=\vbox{\strut}%                       get the height  of a line
\tabvoiddim=\ht\tabvoidbox\advance\tabvoiddim by #1%
\rule{0.0em}{\tabvoiddim}}
\def\botlinespace#1{%
\setbox\tabvoidbox=\vbox{\strut}%                       get the height  of a line
\tabvoiddim=\dp\tabvoidbox\advance\tabvoiddim by #1%
\rule[-\tabvoiddim]{0.0em}{\tabvoiddim}}
%
